\documentclass[10pt, twocolumn]{article}
\usepackage{graphicx} % Required for inserting images

\title{Tutorial}
\author{Jake Moore}
\date{March 2024}

\begin{document}

\maketitle

\section{Introduction}
 My Senior Comps Project is to build a game. I am going to build an open world type game centered around a town/city. It will be built with the use of the free assets on Unreal Engine and the main feature of my comps will be a dungeon. The dungeon will be a "secret" area of the map that the player finds while exploring the town/city. Within the dungeon there will be tougher monsters than outside as well as puzzles to advance through it and a final boss. There will be a reward at the end befitting of the challenge and upon completion the character will have beat the game. Since most of the map will be built using free assets created by others, this dungeon will represent the deep dive section of the game with the puzzles and mini-games being the difficulty of the project.
 The tutorial is relevant to my comps because it is building a simple 3rd person game, which is the same type of game as my comps project. More specifically, the tutorial is about how to create a 3rd person platforming type game, where the player jumps from platform to platform while fighting enemies. The player uses a shotgun and the enemies fight back, can damage the player, and also can die. All of these aspects, both the fighting as well as the platforming will be very useful towards making the dungeon in my comps. The main goal of the tutorial is to follow along and create a simple but functioning game while learning the different main components of the game engine.

 \section{Methods}
 The tutorial is very comprehensive and involves a ton of different processes. I took no deviations from the tutorial, as any deviations would require me to find both other animations and other assets, such as the correct animations for the weapon type and the weapon itself. The first thing I did was to import a basic set of animations into the project from the Unreal Engine Market place, these included an animation for walking, running, idling, and shooting a rifle. I then had to re-target the animations so that they would be compatible with an Unreal 5 mannequin instead of the original Unreal 4 compatibility. Then I created a blend space, which allows for a smooth transition between animations, in this case it was the movement animations. The blend space creates a transition from the idle animation to the jogging animation. I then created an animation montage for firing the weapon, this required creating a variable for speed as well as accessing the character information and finding character velocity. But, velocity is a vector and speed is a float so I had to cast velocity to a float, and then use the velocity to change the animation slightly depending on the movement speed of the character. Then I had to get a free shotgun asset from sketchfab and import it to the project on Unreal. But, when important an asset it doesn't have the textures applied, so I had to edit the shotgun and add the textures to get a better looking gun. I then added the shotgun as a static mesh to my character mannequin. I had to add a socket to the right hand of the character so that the gun would move with the animations of the character. At this point pressing play on the demo gave me the default character, which has standard WASD movement and space as a jump, but now it has the movement and idle animations that I added, as well as carrying a shotgun that moved with the animations. I then had to had a shooting capability for the shotgun. I did this by creating a sphere blueprint, adding the projectile movement component. I then created a new input action for left click creating a sphere and used an arrow to mark the bullet spawn location at the end of the gun. Naturally the arrow is only visible for development and not when running the game. I then had to use get world transformation so that the arrow is the spawn location for the bullets when plugged into a spawn actor function that actually creates the bullet. Within spawn actor I have the option to change the size of the bullet and I made it much smaller than the default so it was more the size of a bullet and changed the collision setting so that even if the gun is shot close to something the bullet still spawns. This lead to a funny bug as when the bullet collides with something it is programmed to stop, was instantly spawning in bullets when I pressed left-click the bullets where instantly hitting each other and stopping at the end of the gun. At this point bullets didn't yet de-spawn so moving and shooting left a trail of bullets in the air that the player could stand on.  To fix this I had to switch the input action to "one click means one action" from "as long as clicking perform action". I then changed the bullet so that on contact it "kills actor" or de-spawns and changed the bullet speed so they lived up to the name bullet. I then changed the view point by editing the Camera boom component since at default the camera is really far from the character and directly behind the character model so you can't really see what you are doing. I made the camera closer and offset to the right to match popular 3rd person shooting games such as Fortnite.

 \section{Metrics and Results}
Simply put, the goal was to build a functioning game, so if the game is functioning you have successfully completed the goal. At the same time this is a really hard goal to measure as a functioning game isn't necessarily the same as the one created by completing the tutorial. But by following these measurements I think I can say that I have partially achieved the goal of the tutorial. I can claim a partial accomplishment of the goal since while I didn't complete the entire tutorial, I created the game (just a simpler version without enemies and separate moving platforms) and also learned a lot. My simpler game takes place on one platform and the player can run and jump around with animations and fire their weapon. I have learned a ton, but the actual building part is far behind what I have learned as it is really complicated to build a game. Currently I have watched the hour and a half tutorial 3 times, so I have a good understanding of what is going on. But it took me 2 ours just to follow along with the first 30 minutes of the tutorial as I am neither familiar with Unreal or fully confident in what I am doing. 
 \section{Reflection}
The only issue I have with the process was in my choice of tutorial, an hour and a half long might not seem that long but the pace of the tutorial is really fast. This isn't really an issue since I can pause it to catch up and rewind when I missed something. But the pacing was so fast that it I spent most of the time with a paused screen trying to catch up to what I just saw. If this video was paced more regularly it would be closer to 3 hours long and I wouldn't have chosen it for the project. That being said I am really happy with the tutorial, including the time I spent watching other tutorials in my search for the one most similar to my comps while also being relatively simple to follow along with, I probably spent upwards of 10 hours on this project not including what I spent on the actual writing part. But, I had a lot of fun doing it, I love playing video games, and I had a lot of fun following along with the tutorial, watching it without following along and even watching parts of other tutorials in the searching process, many of which I have saved to watch later. Currently I have only followed along with a third of the tutorial, but I intend to continue following along over the coming days and finishing the game this weekend. I think I have a great starting point for my comps and while I would have loved to follow the entire tutorial, if I had gone through the tutorial at that speed of the video I would have forgotten most of the information contained in it and just focused on copying what the tutorial did rather than learning what they were doing. As far as the topic goes, I think I chose the topic that I am the most interested in and will enjoy doing the most. But this also raised some concerns as there are so many things that I didn't think about when it comes to making a game so it's going to take much more time than I thought to make the game but I really think that I am going to enjoy doing it so I'm not worried.
\end{document}
